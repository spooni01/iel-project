\section{Příklad 3}
% Jako parametr zadejte skupinu (A-H)
\tretiZadani{G}

%%% Krok 1
\begin{center}
    \textbf{Krok 1} - Prevedieme odpor na vodivosť
\end{center}

\begin{gather*}
    G = \frac{1}{R}
\end{gather*}

%%% Krok 2
\begin{center}
    \textbf{Krok 2} - Vytvoríme si rovnice pre uzly
\end{center}
\begin{gather*}
    A: I_{R_{5}} - G_{5} \times (U_{A} - U_{B}) - U_{A} \times G_{1} - G_{4} \times (U_{A} - U_{B}) = -I_{1}\\
    B: -I_{R_{5}} + G_{5} \times (U_{A} - U_{B}) + G_{4} \times (U_{A} - U_{B}) - G_{3} \times (U_{B} - U_{C}) = -I_{2}\\
    C: G_{3} \times (U_{B} - U_{C}) - U_{C} \times G_{2} = I_{2}\\
\end{gather*}

Rovnice prevedieme do matice a vypočítame ju a zistíme, že:
\begin{gather*}
    U_A = 53.3124V\\
    U_B = -23.9928V \\
    U_C = -20.8676V
\end{gather*}

Vypočítame $\boldsymbol{U_{R_3}}$ a taktiež $\boldsymbol{I_{R_3}}$:
\begin{gather*}
    \boldsymbol{U_{R_3}} = U_B - U_C = \textbf{-3.1252V} \\
    \boldsymbol{I_{R_4}} = \frac{U_{R_3}}{R_3} = \textbf{-0.0590A}
\end{gather*}