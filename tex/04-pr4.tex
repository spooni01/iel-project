\section{Příklad 4}
% Jako parametr zadejte skupinu (A-H)
\ctvrtyZadani{G}

%%% Krok 1
\begin{center}
    \textbf{Krok 1} - Zistíme uhlovú frekvenciu $\omega$
\end{center}

\begin{gather*}
    \omega = 2 \times \pi \times f = 120\pi rad/s
\end{gather*}

%%% Krok 2
\begin{center}
    \textbf{Krok 2} - Zostavíme matice pre smyčky
\end{center}

\begin{gather*}
    \begin{bmatrix} 
	\omega \times L_{1} \times i - R_{1} + \frac{1}{\omega \times C_{2}} \times i & -R_{1} & \frac{1}{\omega \times C_{2}} \times i \\
	R_{1} & \frac{1}{\omega \times C_{1}} + R_{1} & 0\\
	\frac{1}{\omega \times C_{2}} \times i & 0 & \omega \times L_{2} \times i + R_{2} + \frac{1}{\omega \times C_{2}}\\
	\end{bmatrix}
    \times
    \begin{bmatrix}
    I_{A}\\
    I_{B}\\
    I_{C}\\
    \end{bmatrix}
    =
    \begin{bmatrix}
    U_{1}\\
    U_{2}\\
    U_{2}\\
    \end{bmatrix}
	\quad
\end{gather*}

%%% Krok 3
\begin{center}
    \textbf{Krok 3} - Vypočítame determinanty

    Po vyriešení matice vypočítame Cramerovým a Sarrusovým pravidlom determinanty:
\end{center}

\begin{gather*}
    \triangle = -4.7944 + 5.7079i\\
    \triangle_{3} = 5.2874 + 7.9107i\\
    I_{C} = I_{L_{2}} = \frac{\triangle_{3}}{\triangle} = (-0.0089 - 0.0176i)A\\
\end{gather*}


%%% Krok 4
\begin{center}
    \textbf{Krok 4} - Dopočítame napätie $U_{L_{2}}$
\end{center}

\begin{gather*}
    X_{L_{2}} = \omega \times L_{2} \times i = 22.6195i\omega\\
    U_{L_{2}} = X_{L_{2}} \times I_{L_{2}} = (0.3973 - 0.2022i)V\\
\end{gather*}


%%% Krok 5
\begin{center}
    \textbf{Krok 5} - Vypočítame výsledok - $|U_{L_{2}}|$ a $\Phi_{L_{2}}$
\end{center} 

\begin{gather*}
    \boldsymbol{|U_{L_{2}}|} = \sqrt{Re(U_{L_{2}})^{2} + Im(U_{L_{2}})^{2}}=\boldsymbol{0.4458V}\\
    \boldsymbol{\Phi_{L_{2}}} = arctan(\frac{Im(U_{L_{2}})}{Re(U_{L_{2}}}) = \boldsymbol{-0.4707rad}\\
\end{gather*}
